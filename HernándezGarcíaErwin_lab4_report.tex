\documentclass[titlepage, 12pt]{article}

\usepackage[utf8]{inputenc}			%Caracteres
\usepackage{amssymb, amsmath, amsfonts, mathtools}		%Matemáticas
\usepackage[bookmarks,hidelinks]{hyperref}	%Links en el ToC
\usepackage[usenames,dvipsnames]{color}		%Colores
\usepackage{listings}				%Código
\usepackage{url}				%Links web
\usepackage[hypcap]{caption}	%Imagenes
\usepackage{float}				%Imágenes

\usepackage{pdflscape}

%Ajustar márgenes
\topmargin=-0.45in
\evensidemargin=0in
\oddsidemargin=0in
\textwidth=6.5in
\textheight=9.0in
\headsep=0.25in

\numberwithin{equation}{section}%Número de ecuaciones (#Sección, #Ecuación)
\numberwithin{figure}{section}%Número de imágenes (#Sección, #Imagen)
\numberwithin{table}{section}%Número de imágenes (#Sección, #Tabla)

\lstdefinestyle{customc}{
	language=C,
	frame=single, % Single frame around code
	basicstyle=\small\ttfamily, % Use small true type font
	keywordstyle=[1]\color{Blue}\bf, % C functions bold and blue
	keywordstyle=[2]\color{Purple}, % C function arguments purple
	keywordstyle=[3]\color{Blue}\underbar, % Custom functions underlined and blue
	identifierstyle=, % Nothing special about identifiers                                         
	commentstyle=\usefont{T1}{pcr}{m}{sl}\color[rgb]{0.0,0.4,0.0}\small, % Comments small dark green courier font
	stringstyle=\color{Purple}, % Strings are purple
	showstringspaces=false, % Don't put marks in string spaces
	tabsize=2, % 2 spaces per tab
	% Put standard C functions not included in the default language here
	morekeywords={},
	% Put C function parameters here
	morekeywords=[2]{},
	% Put user defined functions here
	morekeywords=[3]{},
	morecomment=[l][\color{Blue}]{...}, % Line continuation (...) like blue comment
	numbers=left, % Line numbers on left
	firstnumber=1, % Line numbers start with line 1
	numberstyle=\tiny\color{Blue}, % Line numbers are blue and small
	stepnumber=5, % Line numbers go in steps of 5
	extendedchars=true,
	inputencoding=utf8,
	literate={á}{{\'a}}1 {é}{{\'e}}1 {í}{{\'i}}1 {ó}{{\'o}}1 {ú}{{\'u}}1 {ñ}{{\~n}}1,
	} 

\lstdefinestyle{customcpp}{
	language=C++,
	frame=single, % Single frame around code
	basicstyle=\small\ttfamily, % Use small true type font
	keywordstyle=[1]\color{Blue}\bf, % C++ functions bold and blue
	keywordstyle=[2]\color{Purple}, % C++ function arguments purple
	keywordstyle=[3]\color{Blue}\underbar, % Custom functions underlined and blue
	identifierstyle=, % Nothing special about identifiers                                         
	commentstyle=\usefont{T1}{pcr}{m}{sl}\color[rgb]{0.0,0.4,0.0}\small, % Comments small dark green courier font
	stringstyle=\color{Purple}, % Strings are purple
	showstringspaces=false, % Don't put marks in string spaces
	tabsize=5, % 5 spaces per tab
	% Put standard C++ functions not included in the default language here
	morekeywords={std,string,cout,endl,ifstream,ofstream,ios},
	% Put C++ function parameters here
	morekeywords=[2]{iostream,fstream},
	% Put user defined functions here
	morekeywords=[3]{},
	morecomment=[l][\color{Blue}]{...}, % Line continuation (...) like blue comment
	numbers=left, % Line numbers on left
	firstnumber=1, % Line numbers start with line 1
	numberstyle=\tiny\color{Blue}, % Line numbers are blue and small
	stepnumber=5, % Line numbers go in steps of 5
	extendedchars=true
	inputencoding=utf8,
	literate={á}{{\'a}}1 {é}{{\'e}}1 {í}{{\'i}}1 {ó}{{\'o}}1 {ú}{{\'u}}1 {ñ}{{\~n}}1,
}

\title{DES}
\author{Eron Romero Argumedo\\Erwin Hernandez Garcia\\3CV1}
\date{September 27, 2016}

\newcommand{\imagen}[4][]{
	\begin{figure}[H]
		\centering
		\includegraphics[#1]{#2}
		\caption{#3}
		#4
	\end{figure}
}

\newcommand{\codescript}[4][]{
	\begin{itemize}
		\item[]\lstinputlisting[caption=#4,label=#3,style=custom#2, #1]{#3.#2}
	\end{itemize}
}

\newcommand{\delimitCodeScript}[6][]{
	\begin{itemize}
		\item[]\lstinputlisting[lastline=#6,firstline=#5, caption=#4,label=#3#5#6,style=custom#2, #1]{#3.#2}
	\end{itemize}	
}

\begin{document}
	\maketitle
	\pagenumbering{roman}
	\tableofcontents
	\listoffigures
	\newpage
	\pagenumbering{arabic}
	\section{Theory}
		\subsection{DES}
		Originally Lucifer was intended to protect the data from a bank. Later the National Bureau of Standards make a petition to create a crypto algorithm to be made a new standard. On 1974 Lucifer was accepted and after work with the NSA, DES was created. Lucifer got modifications to its internal functions and a reduction on the key size from 112 bits to 56 bits.
		
		There were rumours about weakness that NSA had built into the algorithm, but to this day, no evidence has been found. The first DES break was reported in 1997 and it took about three months. Currently the record for breaking DES is 10 hours.
		
		DES is essentially a Feistel network with 16 rounds operating on blocks of 64bits and using a key of 56 bits. DES varies slightly from a Feistel network in the initial permutation that is applied before the first round starts and whose inverse is applied after the last round. It executes the following steps:
		\begin{enumerate}
			\item The right side of the block is expanded from 32bits to 48bits by doubling 16bits and permuting the block.
			\item The exclusive OR of that value with the round key is computed.
			\item The block of 48bits is split into eight blocks with six bit each. Each of them is the input to one of the S-boxes, Each S-box yields a 4bit output.
			\item The permutation P is applied to the 32bits block yielding the round output.\cite{feistel}
		\end{enumerate}
		\imagen[]{Picture2}{Round for DES}{\cite{feistel}\label{DESRound}}
		\subsection{Horst Feistel}
		Horst Feistel was a german cryptographer that worked designed ciphers at IBM, there he started the investigation that finished with the creation of DES. He was one of the first non-governmental cryptographers that studied the design and theory of block ciphers.
		
		While working at IBM, he lead the design of a cipher called ``Lucifer'' and DES.\cite{feistelbiography}
		\subsection{Feistel Network}
		A Feistel network is a cipher scheme created by Horst Feistel during his work on the cipher Lucifer for IBM. It defines a function $\text{F}: \{0,1\}^{2n} \rightarrow\{0,1\}^{2n}$. It depends on the number of rounds $d \in \mathbb{N}$ and the round functions $f_{i}, ..., f_{d}: \{0,1\}^{2n} \rightarrow\{0,1\}^{2n}$
		
		The F functions is defined as follows: It works on d rounds, the i-th round does:
		\imagen[]{Picture1}{Round for a Feistel Network}{\label{FeistelRound}}
		\begin{enumerate}
			\item The input is the previous round output.
			\item Split by two the input: $L_{i-1}, R_{i-1}$.
			\item Applies $f_i$ to the right part yielding $f_{i}(R_{i-1})$.
			\item Calculate $L_{i-1} \oplus f_{i}(R_{i-1})$, the result will be the right side of the output.
			\item The left side of the output is the right side of the input.
		\end{enumerate}\cite{feistel}
		\subsection{Permutation}
		\delimitCodeScript{c}{des}{DES'Permutations}{69}{94}
		The i-th bit in the permutation represents the position from where the i-th bit in the message is taken. That is, for the first bit we shall take the 58th bit, for the second we shall take the 50th bit and so on.
		\imagen[width=\linewidth]{DES-ip}{Initial Permutation}{\cite{DESIP}\label{IP}}
		\imagen[width=\linewidth]{DES-ip-1}{Inverse of the initial permutation}{\cite{DESIP-1}\label{IP-1}}
		\imagen[width=\linewidth]{DES-pp}{Permutation P}{\cite{DESP}\label{P}}
		\subsection{S-box}
		The S-box is a table with 4 rows and 16 columns. Each S-box replaces a 6-bit input with a 4-bit output. Given a 6-bit input, the 4-bit output is found by selecting the row using the outer two bits, and the column using the inner four bits. For example, the input "011011" has outer bits "01" (second row) and inner bits "1101" (14th column). The corresponding output for S-box S5 would be "1001", the value in the second row, 14th column.
		\begin{landscape}
			\delimitCodeScript{c}{des}{DES' S-boxes}{36}{67}
		\end{landscape}
		\subsection{Weak and Semi-weak keys}
		There are certain keys that makes DES encrypt operation behave as decryption.
		\begin{itemize}
			\item Alternating zeros and ones
			\item Alternating F and E
			\item 0xE0E0E0E0F1F1F1F1
			\item 0x1F1F1F1F0E0E0E0E
		\end{itemize}
		If the implementation doesn't account for parity bits, then
		\begin{itemize}
			\item 0x0000000000000000
			\item 0xFFFFFFFFFFFFFFFF
			\item 0xE1E1E1E1F0F0F0F0
			\item 0x1E1E1E1E0F0F0F0F
		\end{itemize}
		When using weak keys, the generated subkeys are identical, which, as DES being a Feistel network, it makes the odd rounds encrypt while the even decrypt.
		
		The semi-weak keys will only generate two different subkeys. This keys exist in pairs, with the characteristic that:
		\begin{equation*}
		E_{K_1}(E_{K_2}(M))=M
		\end{equation*}
		\begin{itemize}
			\item 0x011F011F010E010E y 0x1F011F010E010E01
			\item 0x01E001E001F101F1 y 0xE001E001F101F101
			\item 0x01FE01FE01FE01FE y 0xFE01FE01FE01FE01
			\item 0x1FE01FE00EF10EF1 y 0xE01FE01FF10EF10E
			\item 0x1FFE1FFE0EFE0EFE y 0xFE1FFE1FFE0EFE0E
			\item 0xE0FEE0FEF1FEF1FE y 0xFEE0FEE0FEF1FEF1
		\end{itemize}
		\subsection{3DES}
		Triple DES consists in applying the following to encrypt:
		\begin{equation*}
			E_{3DES}(K_1, K_2, K_3, m) = E_{DES}(K_3, D_{DES}(K_2, E_{DES}(K_1, m)))
		\end{equation*}
		and to decrypt:
		\begin{equation*}
			D_{3DES}(K_1, K_2, K_3, m) = D_{DES}(K_1, E_{DES}(K_2, D_{DES}(K_3, m)))
		\end{equation*}
		\cite{feistel}
	\begin{landscape}
			\section{Functions}
			\delimitCodeScript{c}{TripleDES/TripleDES}{Key generation}{77}{101}
			The key generation creates three keys that will be used by Triple DES, then they are saved in a file with a name specified by the user.
			\linebreak
			\delimitCodeScript{c}{TripleDES/TripleDES}{Encrypt/Decrypt}{106}{226}
			The function to encrypt/decrypt does the action specified by the user, the user must give the name of the file containing the key, the name of the file containing the cipher text or the name of the file containing the plaint text  and the name of the output file.
	\end{landscape}
		
	\section{Screenshots}
		\subsection{Usage}
		\imagen[width=\linewidth]{Usage.png}{Usage of the program}{\label{Usage}}
		\subsection{Results}
		\imagen[width=\linewidth]{PlainText.png}{Original plaintext}{\label{Plaintext}}
		\imagen[width=\linewidth]{Encrypt.png}{Ciphertext obtained after encrypt with Triple DES}{\label{Ciphertext1}}
		\imagen[width=\linewidth]{Decrypt.png}{Plaintext obtained after decrypt with Triple DES}{\label{Ciphertext2}}	
	\section{Signature}
	\newpage
	\imagen[width=\linewidth]{Firma1}{Eron Romero Argumedo}{\label{Eron}}
	\imagen[height=\textheight]{Firma2}{Erwin Hernández García}{\label{Erwin}}
	\bibliographystyle{IEEEtran}
	\bibliography{Bibliography}
\end{document}